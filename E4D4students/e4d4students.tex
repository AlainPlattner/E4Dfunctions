\documentclass[11pt]{article}

\usepackage{url}
\setlength{\parskip}{0.5cm plus4mm minus3mm}

\setlength{\parindent}{0cm}

\textwidth=6.4in
\textheight=8.5in
\hoffset=-0.7in
\voffset=-0.7in

\title{E4D tutorial}
\author{Alain Plattner}


\begin{document}

\maketitle

\section{Introduction}
E4D is a powerful electrical resistivity tomography program written by
Tim Johnson at PNNL. It can be freely downloaded and installed from
\url{https://e4d.pnnl.gov/}. The program's strength are that it is
parallelized and can therefore solve large problems (many inversion
cells) very quickly. The disadvantage of E4D is that it is harder to
learn than for example BERT.

E4D uses several types of configuration files which we will discuss in the
following. The Matlab/Octave scripts that come with this tutorial will
be helpful to generate the content for many of these input files.

In this tutorial we will go step by step, starting with generating a
simple mesh and moving through calculating inversions. Preparing the
input files for complex meshes is the hardest part of using E4D, so we
will go step by step from simple (but usually useless) meshes to
complicated meshes.

Generally the work flow for using E4D will be to first construct a
mesh, and then run the inversion on a prepared mesh. E4D always
requires an input file named \verb+e4d.inp+. In this input file, the
user specifies what needs to be done (e.g. mesh generation,
inversion), and the names of the configuration and data files.


\section{Mesh generation}
On your desktop, or in any folder you would like to work in, create a
file named \verb+e4d.inp+. You can do that for example with emacs by
running in your folder

\qquad \verb+emacs e4d.inp+ 


\subsection{My first mesh}\label{simplemesh}

To make e4d create a mesh, the first line of \verb+e4d.inp+ needs to
be the number 1. The second line needs to be the name of our mesh
configuration file. Let's call it \verb+myfirstmesh.cfg+. So make your
\verb+e4d.inp+ text file look like

\verb+1+\\
\verb+myfirstmesh.cfg+

Of course, we now need to create the mesh configuration file
\verb+myfirstmesh.cfg+. To do that, create it again using emacs

\qquad \verb+emacs myfirstmesh.cfg+

This file will now be a bit more complicated than the input file.
A mesh file will always consist of the following components:
\begin{itemize}
\item The mesh quality (how deformed can the tetrahedra be), and
  maximum tetrahedra volume
\item the bottom elevation of the mesh
\item some information on how to build the mesh (this will always be
  the same so don't worry about this now)
\item The points around which the mesh will be generated including
  electrode positions, topography, boundary points for the internal
  and external zones, electrode depth points for mesh refinement
  around the electrodes, etc.
\item Internal planes to delineate between mesh zones
\item holes (I have not yet worked with holes in the mesh so we will
  skip this in this tutorial)
\item zone descriptions
\item information about writing a paraview/exodus file
\end{itemize}

Let's start with the simplest possible configuration. We first need to
learn how we write points. A point always has the form

\emph{pointnumber xcoord ycoord zcoord type}

\emph{type} can be subsurface point (0), or surface point (1) or
external boundary point (2).

The external boundary points define the outside boundary of our region
so we will always need them.

Let's write our \verb+myfirstmesh.cfg+ file. Let's make the mesh
quality 1.3 and the maximum volume $10^{12}$. The surface will be
defined by the external boundary points (which by definition are on
the surface), and the bottom will be at -20 m. We want the external
boundary to be at at four points (-10/-30/0), (50/-5/1), (40/70/-1.2),
and (-20/60/0). The units of these coordinates are meters.

\newpage
So the lines in our file \verb+myfirstmesh.cfg+ will be:

\verb+1.3 1e12   (mesh quality, max volume)+\\
\verb++\\
\verb+-20   (bottom elevation of mesh)+\\
\verb+1+\\
\verb+"tetgen"+\\
\verb+"triangle"+\\
\verb++\\
\verb+4   (number of points)+\\
\verb+1 -10 -30 0 2   (first boundary point)+\\
\verb+2 50 -5 1 2  (second boundary point)+\\
\verb+3 40 70 -1.2 2  (etc)+\\
\verb+4 -20 60 0 2+\\
\verb++\\
\verb+0   (no internal planes)+\\
\verb++\\
\verb+0   (no holes)+\\
\verb++\\
\verb+1   (one zone)+\\
\verb+1 0 0 -2 50 0.01   (point in zone, max mesh size this zone, conductivity)+\\
\verb++\\
\verb+1   (save mesh as exodus file)+\\
\verb+"bx"  (use the program "bx")+\\
\verb++\\
\verb+1   (mesh translation option)+


\vspace{2cm}

If you want you can omit the comments in parenthesis. These will not
be used by the program but it may make it easier for you to edit the
mesh configuration file.

In this file we needed to define parameters such as maximum tetrahedra
volume for each zone (at this point we only have one zone), and the
conductivity. To do this we needed to pick a random point within our
zone, we picked (0/0/-2), and set the maximum tetrahedra volume, we
selected 50 m$^3$, and for the conductivity, we selected 0.01
S/m. These values are for zone 1, therefore the line read

\emph{1 0 0 -2 50 0.01}

To create the mesh, run in your command line

\qquad \verb+e4d+

The program will automatically look in the input file \verb#e4d.inp#
and use the mesh configuration file you specified there and will
create a lot of auxiliary files. Open the resulting file
\verb#myfirstmesh.exo# using paraview.


\textbf{Exercise:}
Play around with the maximum mesh size for the zone (which we had originally set to 5) and see if you can create a finer mesh.


\subsection{Adding electrodes}

The mesh we created in Section~\ref{simplemesh} looks great but will
be of limited use when we want to start running inversions. The
minimal thing that we need to do is to add nodes where we have
electrodes.

Let's assume the simple case where we have only 3 electrodes, all on
the surface, at the points (0/0/0), (0/1/0.5), and (0/2/0.2). We will
add these points to the mesh simply by including them in our list of
points. Since our electrodes are all on the surface, their \emph{type}
has the value 1. Our new list of points therefore looks like

\verb+7   (number of points)+\\
\verb+1 -10 -30 0 2   (first boundary point)+\\
\verb+2 50 -5 1 2  (second boundary point)+\\
\verb+3 40 70 -1.2 2  (etc)+\\
\verb+4 -20 60 0 2+\\
\verb++\\
\verb+5 0 0 0 1   (first electrode)+\\
\verb+6 0 1 0.5 1  (second electrode)+\\
\verb+7 0 2 0.2 1  (third electrode)+

The rest of the file \verb+myfirstmesh.cfg+ stays the same as
before. You do not have to insert an empty line between the external
boundary points and the electrodes, but it may make it easier when you
have many electrodes. 

Run in your command prompt

\qquad \verb+e4d+

and open the resulting mesh again with paraview. You will notice that
the mesh looks a bit different. There are smaller tetrahedra where the
electrodes are.

When running electrical resistivity inversions, you will likely run
into problems when the mesh is too coarse around your electrodes. To
create a fine mesh around the electrodes, we will add depth points
underneath the electrodes. These depth points will have the same x and
y coordinates as the electrodes but are a few cm (or m) below the
electrodes (the depth will depend on how fine you will want the mesh
around your electrodes). Let's pick 1 cm in this example (our
electrodes are only 1 m apart). Because
these points are below the surface, their \emph{type} is 0. So your
new list of points will be

\verb+10   (number of points)+\\
\verb+1 -10 -30 0 2   (first boundary point)+\\
\verb+2 50 -5 1 2  (second boundary point)+\\
\verb+3 40 70 -1.2 2  (etc)+\\
\verb+4 -20 60 0 2+\\
\verb++\\
\verb+5 0 0 0 1   (first electrode)+\\
\verb+6 0 1 0.5 1  (second electrode)+\\
\verb+7 0 2 0.2 1  (third electrode)+\\
\verb++\\
\verb+8 0 0 -0.01 0   (first electrode depth point)+\\
\verb+9 0 1 0.49 0  (second electrode depth point)+\\
\verb+10 0 2 0.19 0  (third electrode depth point)+\\

When you run

\qquad \verb+e4d+

with this new \verb+myfirstmesh.cfg+ file, and you look at the result
using paraview, you will see that the mesh around the electrodes is
very fine.

Congratulations, you created your first mesh that could be used for
electrical resistivity tomography. However, to get good results with
reasonable computing power, there is more to learn.

\subsection{Internal boundaries}

One of the problems of electrical resistivity tomography is that our
mesh ends somewhere. This of course is not the case in nature. The
world does not just end. This boundary is therefore artificial.

To avoid problems with this artificial boundary, we will usually have
the external boundary far away, by placing the external boundary
points (the \emph{type} 2 points) far away.

But this leads to the problem that we either can only have a coarse
grid away from the electrodes, or we need to fill the entire volume
with a fine mesh.

Luckily, E4D allows us to create an internal zone with fine mesh
(where we have the electrodes and good coverage), and an external
zone with coarse mesh (which allows us to push the external boundary
far away). In fact, we can have as many zones as we want. We could
make a very fine mesh zone very close to the electrodes, then a
coarser mesh zone a bit further away, and finally a very course mesh
zone surrounding everything.

In this example we will only create two zones, an external zone and an
internal zone, but you can use what you learned here to create more
complex cases.

To create zones we need to create zone boundaries and to create zone
boundaries we need to first pick zone boundary points.

Let's say we want the inner-zone boundary points (-2/-2/-0.3),
(-2/4/0.2), (2/4/0), and (2/-2/0.1). \textbf{Note that these points
  are in clockwise order}. It is important that you enter your points
in clockwise (or counter-clockwise) order, otherwise the boundary
generation will not work.

Writing your own internal boundaries can be difficult. To make life
easier, I wrote the \textsc{Matlab}/\textsc{Octave} program
\verb+makezone.m+ to do this for us. It's in the folder \verb+m-files+
in the same repository where you found this tutorial. Run, in
\textsc{Matlab} or \textsc{Octave}

\qquad \verb+>> help makezone+

to see what input is required. The simplest way of using it is to give
it the surface points (or shallower subsurface points, if you want a completely
buried zone), the depth of the zone under the surface (let's say we
want the internal zone to end at 5 m depth), and the number of
the first point (we already have 10 points in our list, so the first
point will be 11).  Let's say we want the text output to be stored in
the file \verb+myfirstzone.txt+. To do all of this, run in
\textsc{Matlab} or \textsc{Octave}

\qquad \verb+>> makezone([-2,-2,2,2],[-2,4,4,-2],[-0.3,0.2,0,0.1],-5,11,'myfirstzone.txt')+


The resulting file \verb+myfirstzone.txt+ will contain the points
with point numbers between 11 and 18, and information about internal
planes. Add the points to your list of points in your
\verb+myfirstmesh.cfg+ file (don't forget to update the total number
of points) and replace the line saying that there are
no internal planes with the internal planes you generated.

So the points part of your file \verb+myfirstmesh.cfg+ will now be


\verb+18   (number of points)+\\
\verb+1 -10 -30 0 2   (first boundary point)+\\
\verb+2 50 -5 1 2  (second boundary point)+\\
\verb+3 40 70 -1.2 2  (etc)+\\
\verb+4 -20 60 0 2+\\
\verb++\\
\verb+5 0 0 0 1   (first electrode)+\\
\verb+6 0 1 0.5 1  (second electrode)+\\
\verb+7 0 2 0.2 1  (third electrode)+\\
\verb++\\
\verb+8 0 0 -0.01 0   (first electrode depth point)+\\
\verb+9 0 1 0.49 0  (second electrode depth point)+\\
\verb+10 0 2 0.19 0  (third electrode depth point)+\\
\verb++\\
\verb+11 -2.000000 -2.000000 -0.300000 1 +\\
\verb+12 -2.000000 4.000000 0.200000 1 +\\
\verb+13 2.000000 4.000000 0.000000 1 +\\
\verb+14 2.000000 -2.000000 0.100000 1 +\\
\verb++\\
\verb+15 -2.000000 -2.000000 -5.000000 0 +\\
\verb+16 -2.000000 4.000000 -5.000000 0 +\\
\verb+17 2.000000 4.000000 -5.000000 0 +\\
\verb+18 2.000000 -2.000000 -5.000000 0 +\\

And the internal planes part will now be

\verb+5 number of internal planes+\\
\verb+4 10+\\
\verb+11 12 16 15+\\
\verb+4 10+\\
\verb+12 13 17 16+\\
\verb+4 10+\\
\verb+13 14 18 17+\\
\verb+4 10+\\
\verb+11 14 18 15+\\
\verb+4 10+\\
\verb+15 16 17 18+\\ 
\verb++\\

Note that here we used the internal boundary number 10, which I set as
default. If you have several internal zones, then you may want to use
different internal boundary numbers (one for each zone).


We will also need to add that we have a second zone. We do that by
simply adding a line in the zones part of the file
\verb+myfirstmesh.cfg+ and set the maximum mesh volume and
conductivity for this zone. For this we need to find a point in the
external zone. In our case, the point (10/10/-10) should do. The point
(0/0/-2) is still within our internal zone. Let's set, for the
internal zone, the maximum mesh volume 1 m$^3$ and for the external zone
the maximum mesh volume $10^5$ m$^3$. We set 0.01 S/m for the
conductivity of both zones.

Hence our new ``zones'' part of the file \verb+myfirstmesh.cfg+ will
be

\verb+2  (two zones)+\\
\verb+1 0 0 -2 1 0.01  (internal zone with fine mesh)+\\
\verb+2 10 10 -10 1e5 0.01  (external zone with coarse mesh)+


After updating your file \verb+myfirstmesh.cfg+, run on the command
line

\qquad \verb+e4d+

and look at the new mesh using paraview.

You will see that your new mesh has two different zones and the mesh
of the internal zone is much finer than the mesh of the external zone.


Congratulations, you now have the knowledge needed to create
electrical resistivity tomography meshes for quite general settings.

\textbf{A note on topography:} As you noticed, all of our points had a
z-axis value. This is how you define topography. As long as your point
is either of \emph{type} 1 or 2, the mesh generator will know that it
is on the surface and will incorporate it into the topography. To
incorporate complex topography, simply add all the topography points
to your points list as \emph{type} 1 surface points (don't forget to
update the total number of points).

\textbf{Exercise:} Instead of a 4 point internal zone boundary, make a
5 point internal zone boundary. Make sure you enter the points in
clockwise orientation.

\textbf{Exercise:} Download topography points from an online database
(for example \url{http://opentopo.sdsc.edu/datasets}) and create a
mesh for these topography points.

To make your life easier I wrote \textsc{Matlab}/\textsc{Octave}
functions that can help with some of the tasks for mesh
generation. You can find them in the folder \verb+m-files+. For
example, the function \verb+elecs2points.m+ will write electrode
coordinates in the format that can directly be copy-pasted into your
e4d mesh configuration file. The function \verb+topodata2grid.m+ can
turn ASCII point-cloud data downloaded from
\url{http://opentopo.sdsc.edu/datasets} into the format that can be
copy-pasted into the e4d mesh configuration file (including
downsampling).





\end{document}


