\documentclass[11pt]{article}

\usepackage{url}
\setlength{\parskip}{0.5cm plus4mm minus3mm}

\setlength{\parindent}{0cm}

\textwidth=6.4in
\textheight=8.5in
\hoffset=-0.7in
\voffset=-0.7in

\title{E4D tutorial}
\author{Alain Plattner}


\begin{document}

\maketitle

\section{Introduction}
E4D is a powerful electrical resistivity tomography program written by
Tim Johnson at PNNL. It can be freely downloaded and installed from
\url{https://e4d.pnnl.gov/}. The program's strength are that it is
parallelized and can therefore solve large problems (many inversion
cells) very quickly. The disadvantage of E4D is that it is harder to
learn than for example BERT.

E4D uses several types of configuration files which we will discuss in the
following. The Matlab/Octave scripts that come with this tutorial will
be helpful to generate the content for many of these input files.

In this tutorial we will go step by step, starting with generating a
simple mesh and moving through calculating inversions. Preparing the
input files for complex meshes is the hardest part of using E4D, so we
will go step by step from simple (but usually useless) meshes to
complicated meshes.

Generally the work flow for using E4D will be to first construct a
mesh, and then run the inversion on a prepared mesh. E4D always
requires an input file named \verb+e4d.inp+. In this input file, the
user specifies what needs to be done (e.g. mesh generation,
inversion), and the names of the configuration and data files.


\section{Mesh generation}
On your desktop, or in any folder you would like to work in, create a
file named \verb+e4d.inp+. You can do that for example with emacs by
running in your folder

\qquad \verb+emacs e4d.inp+ 


To make e4d create a mesh, the first line of \verb+e4d.inp+ needs to
be the number 1. The second line needs to be the name of our mesh
configuration file. Let's call it \verb+myfirstmesh.cfg+. So make your
\verb+e4d.inp+ text file look like

\verb+1+\\
\verb+myfirstmesh.cfg+

Of course, we now need to create the mesh configuration file
\verb+myfirstmesh.cfg+. To do that, create it again using emacs

\qquad \verb+emacs myfirstmesh.cfg+

This file will now be a bit more complicated than the input file.
A mesh file will always consist of the following components:
\begin{itemize}
\item The mesh quality (how deformed can the tetrahedra be), and
  maximum tetrahedra volume
\item the bottom elevation of the mesh
\item some information on how to build the mesh (this will always be
  the same so don't worry about this now)
\item The points around which the mesh will be generated including
  electrode positions, topography, boundary points for the internal
  and external zones, electrode depth points for mesh refinement
  around the electrodes, etc.
\item Internal planes to delineate between mesh zones
\item holes (I have not yet worked with holes in the mesh so we will
  skip this in this tutorial)
\item zone descriptions
\item information about writing a paraview/exodus file
\end{itemize}

Let's start with the simplest possible configuration. We first need to
learn how we write points. A point always has the form

\emph{pointnumber xcoord ycoord zcoord type}

\emph{type} can be subsurface point (0), or surface point (1) or
external boundary point (2).

The external boundary points define the outside boundary of our region
so we will always need them.

Let's write our \verb+myfirstmesh.cfg+ file. Let's make the mesh
quality 1.3 and the maximum volume $10^{12}$. The surface will be
defined by the external boundary points (which by definition are on
the surface), and the bottom will be at -10. We want the external
boundary to be at at four points (-1/-3/0), (5/-0.5/1), (4/7/-1.2),
and (-2/6/0).

\newpage
So the lines in our file \verb+myfirstmesh.cfg+ will be:

\verb+1.3 1e12   (mesh quality, max volume)+\\
\verb++\\
\verb+-10   (bottom elevation of mesh)+\\
\verb+1+\\
\verb+"tetgen"+\\
\verb+"triangle"+\\
\verb++\\
\verb+4   (number of points)+\\
\verb+1 -1 -3 0 2   (first boundary point)+\\
\verb+2 5 -0.5 1 2  (second boundary point)+\\
\verb+3 4 7 -1.2 2  (etc)+\\
\verb+4 -2 6 0 2+\\
\verb++\\
\verb+0   (no internal planes)+
\verb++\\
\verb+0   (no holes)+\\
\verb++\\
\verb+1   (one zone)+\\
\verb+1 0 0 -2 5 0.01   (point in zone, max mesh size this zone, conductivity)+\\
\verb++\\
\verb+1   (save mesh as exodus file)+\\
\verb+"bx"  (use the program "bx")+\\
\verb++\\
\verb+1   (mesh translation option)+


\vspace{2cm}

If you want you can omit the comments in parenthesis. These will not
be used by the program but it may make it easier for you to edit the
mesh configuration file.

In this file we needed to define parameters such as maximum tetrahedra
volume for each zone (at this point we only have one zone), and the
conductivity. To do this we needed to pick a random point within our
zone, we picked (0/0/-2), and set the maximum tetrahedra volume, we
selected 5 m, and the conductivity, we selected 0.01 S/m. These values
are for zone 1, therefore the line read

\emph(1 0 0 -2 5 0.01)

To create the mesh, run in your command line

\qquad \verb+e4d+

The program will automatically look in the input file \verb#e4d.inp#
and use the mesh configuration file you specified there and will
create a lot of auxiliary files. Open the resulting file
\verb#myfirstmesh.exo# using paraview.


\textbf{Exercise:}
Play around with the maximum mesh size for the zone and see if you can
create a finer mesh.



NEXT: PUT IN ELECTRODES
THEN: TO REFINE MESH AROUND ELECTRODES, PUT IN POINTS BELOW ELECTRODES.


\end{document}


